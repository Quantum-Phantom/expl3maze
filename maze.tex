\documentclass{ltxdoc}
\title{The \textsf{maze} package\footnote{~~This project is distributed under the \LaTeX~Project Public License, version 1.3c.}}
\usepackage{graphicx}
\usepackage{fontspec}
\setmainfont{Times New Roman}
\usepackage[misc]{ifsym}
\author{Sicheng Du\thanks{~~\Letter~~\href{mailto:siddsc@foxmail.com}{siddsc@foxmail.com}}}
\date{\today~~~~v1.1}
\usepackage[colorlinks,linkcolor=purple]{hyperref}
\usepackage{maze}
\begin{document}
\maketitle
The \textsf{maze} package can generate random square mazes of a specified size. You need to start from the bottom-left corner and reach the top-right corner to play it.
\begin{macro}{\maze}
\marg{size}\oarg{seed} is the syntax of the command that generates a maze. Thereinto
\begin{description}
\item[\marg{size}] controls the density of the walls inside the maze and directly influences its complexity. It must be a positive integer greater or equal to 2.

To have the package produce a satisfactory outcome, it is recommended to input a number between 20 and 30 into \marg{text}. Over large numbers may cause \TeX~to exhaust its capacity and fail to produce anything.
\item[\oarg{seed}] is an optional parameter that specifies the seed for random numbers. If it is omitted, the current time (minute) will be used as the seed instead.
\end{description}\end{macro}

As an example, the following can be created by \cs{maze{30}[4]} and \cs{maze{20}[6]} respectively.
\begin{figure}[h]
\begin{minipage}{.5\textwidth}
\maze{30}[4]
\end{minipage}\hfill
\begin{minipage}{.5\textwidth}
\maze{20}[6]
\end{minipage}
\end{figure}
\eject
\end{document}
