\documentclass{ltxdoc}
\title{The \textsf{maze} package\footnote{~~This project is distributed under the \LaTeX~Project Public License, version 1.3c.}}
\usepackage[margin=2cm]{geometry}
\usepackage{graphicx}
\usepackage{listings}
\lstset{
    basicstyle=\fontspec{Consolas},numbers=left,
    numberstyle=\small,stepnumber=1,numbersep=5pt
}
\usepackage{xcolor}
\usepackage{fontspec}
\setmainfont{Times New Roman}
\usepackage{indentfirst}
\usepackage[misc]{ifsym}
\author{Sicheng Du\thanks{~~\Letter~~\href{mailto:siddsc@foxmail.com}{siddsc@foxmail.com}}}
\date{\today~~~~v1.2}
\usepackage[colorlinks,linkcolor=purple]{hyperref}
\usepackage{maze}
\begin{document}
\maketitle
\section{Changes in this version}
Thanks to the \TeX nicians who kindly gave valuable and earnest suggestions to this package, the version 1.2 is now released. The main changes include
\begin{enumerate}
\item Corrected some mistakes in this manual;
\item Modified the format in the source code to improve compatibility;
\item Improved the output map of the maze to make it clearer.
\end{enumerate}
\section{User instructions}
The \textsf{maze} package can generate random square mazes of a specified size. You need to start from the bottom-left corner and reach the top-right corner to play it.
\begin{macro}{\maze}
\marg{size}\oarg{seed} is the syntax of the command that generates a maze. Thereinto
\begin{description}
\item[\marg{size}] controls the density of the walls inside the maze and directly influences its complexity. It must be a positive integer in the range \textbf{[2,100]}.

To have the package produce a satisfactory output, it is recommended to input a number between 20 and 50 into \marg{text}. Over large numbers may cause \TeX~to exhaust its capacity and fail to produce anything.
\item[\oarg{seed}] is an optional parameter that specifies the seed for random numbers. If it is omitted, the current time (minute) will be used as the seed instead.
\end{description}\end{macro}

As an example, the mazes in Figure \ref{fig:mazes} can be created by \cs{maze\{30\}[4]}, \cs{maze\{20\}[7]} and \cs{maze\{25\}}\footnote{This maze is likely to look different because of the difference of compiling time.} respectively.
\begin{figure}[h]
\centering
\maze{30}[4]\hfill\maze{20}[7]\hfill\maze{25}
\caption{Examples of mazes}\label{fig:mazes}
\end{figure}
\clearpage
\section{Code implementation}
This package uses the \textsf{expl3} programming layer. Under the scope of \cs{ExplSyntaxOn} we first define some variables
\begin{lstlisting}[firstnumber=7]
\int_new:N\l_maze_rand_int                           % the random variable
\int_new:N\l_maze_old_int \int_new:N\l_maze_new_int
\dim_const:Nn\g_maze_size_dim{\linewidth}            % store the line width
\intarray_new:Nn\g_maze_map_intarray{10000}          % map of the maze
\intarray_new:Nn\g_walls_v_intarray{9900}            % existence of vertical
\intarray_new:Nn\g_walls_h_intarray{9900}            % /horizontal walls
\end{lstlisting}

The internal command is defined as \cs{m@ze}. Starting with variable initialization,
\begin{lstlisting}[firstnumber=last]
\newcommand{\m@ze}[2]{
  \sys_gset_rand_seed:n{#2}
  \intarray_gzero:N\g_maze_map_intarray
  \intarray_gzero:N\g_walls_v_intarray
  \intarray_gzero:N\g_walls_h_intarray
  \int_step_inline:nn{#1*#1}{
    \intarray_gset:Nnn\g_maze_map_intarray{##1}{##1}
  }
  \int_step_inline:nn{#1*(#1-1)}{
    \intarray_gset:Nnn\g_walls_v_intarray{##1}{1}
    \intarray_gset:Nnn\g_walls_h_intarray{##1}{1}
  }
\end{lstlisting}

Then we apply the \textit{Kruskal} algorithm to the intarray-based variant of the \textit{Union-find set}. Our loop should end when the beginning and finishing cells are connected. (so that a path exists)
\begin{lstlisting}[firstnumber=last]
  \bool_do_until:nn{
    \int_compare_p:nNn{\intarray_item:Nn\g_maze_map_intarray{1}}
    ={\intarray_item:Nn\g_maze_map_intarray{#1*#1}}
  }{
    \int_set:Nn\l_maze_rand_int{\int_rand:n{#1*(#1-1)}}
    \int_compare:nNnTF{0}={\intarray_item:Nn\g_walls_v_intarray{\l_maze_rand_int}}{}{
      \int_compare:nNnTF{
        \intarray_item:Nn\g_maze_map_intarray{
          \l_maze_rand_int+\int_div_truncate:nn{\l_maze_rand_int-1}{#1-1}
        }
      }={
        \intarray_item:Nn\g_maze_map_intarray{
          1+\l_maze_rand_int+\int_div_truncate:nn{\l_maze_rand_int-1}{#1-1}
        }
        }{}{
        \int_set:Nn\l_maze_new_int{
          \intarray_item:Nn\g_maze_map_intarray{
            1+\l_maze_rand_int+\int_div_truncate:nn{\l_maze_rand_int-1}{#1-1}
          }
        }
        \int_set:Nn\l_maze_old_int{
          \intarray_item:Nn\g_maze_map_intarray{
            \l_maze_rand_int+\int_div_truncate:nn{\l_maze_rand_int-1}{#1-1}
          }
        }
        \intarray_gset:Nnn\g_walls_v_intarray{\l_maze_rand_int}{0}
        \int_step_inline:nn{#1*#1}{
          \int_compare:nNnTF{\l_maze_old_int}={\intarray_item:Nn\g_maze_map_intarray{##1}}
          {\intarray_gset:Nnn\g_maze_map_intarray{##1}{\l_maze_new_int}}{}
        }
      }
    }
    \int_set:Nn\l_maze_rand_int{\int_rand:n{#1*(#1-1)}}
    \int_compare:nNnTF{0}={\intarray_item:Nn\g_walls_h_intarray{\l_maze_rand_int}}{}{
      \int_compare:nNnTF{\intarray_item:Nn\g_maze_map_intarray{\l_maze_rand_int}}
      ={\intarray_item:Nn\g_maze_map_intarray{#1+\l_maze_rand_int}}{}{
        \int_set:Nn\l_maze_new_int{
          \intarray_item:Nn\g_maze_map_intarray{#1+\l_maze_rand_int}
        }
        \int_set:Nn\l_maze_old_int{
          \intarray_item:Nn\g_maze_map_intarray{\l_maze_rand_int}
        }
        \intarray_gset:Nnn\g_walls_h_intarray{\l_maze_rand_int}{0}
        \int_step_inline:nn{#1*#1}{
          \int_compare:nNnTF{\l_maze_old_int}={\intarray_item:Nn\g_maze_map_intarray{##1}}
          {\intarray_gset:Nnn\g_maze_map_intarray{##1}{\l_maze_new_int}}{}
        }
      }
    }
  }
}
\end{lstlisting}

Lastly, we finish off by defining the user command, which outputs a map of the maze. To set the size and draw the boundaries,
\begin{lstlisting}[firstnumber=last]
\NewDocumentCommand\maze{mO{\c_sys_minute_int}}{
  \m@ze{#1}{#2}
  \setlength{\unitlength}{\fp_eval:n{.4/#1}\g_maze_size_dim}
  \begin{picture}(#1,#1)(0,0)
  \put(0,0){\line(0,1){#1}} \put(#1,#1){\line(0,-1){#1}}
  \put(\int_eval:n{#1-1},#1){\line(-1,0){\int_eval:n{#1-1}}}
  \put(1,0){\line(1,0){\int_eval:n{#1-1}}}
\end{lstlisting}

We extract from \cs{g\_walls\_h\_intarray} and \cs{g\_walls\_v\_intarray} and draw a line wherever a wall exists.
\begin{lstlisting}[firstnumber=last]
  \int_step_inline:nn{#1*(#1-1)}{
    \int_compare:nNnTF{0}={\intarray_item:Nn\g_walls_h_intarray{##1}}{}{
      \put(
        \int_mod:nn{##1-1}{#1},
        \int_eval:n{1+\int_div_truncate:nn{##1-1}{#1}}
      ){\line(1,0){1}}
    }
    \int_compare:nNnTF{0}={\intarray_item:Nn\g_walls_v_intarray{##1}}{}{
      \put(
        \int_mod:nn{##1}{#1-1},
        \int_div_truncate:nn{##1-1}{#1-1}
      ){\line(0,1){1}}
    }
  }
  \end{picture}
}
\ExplSyntaxOff
% End of package code
\end{lstlisting}
\end{document}
